% !TEX root = Projektdokumentation.tex

\begin{acronym}[Responsive DesignW]
	\acro{CMS}{\textbf{C}ontent \textbf{M}anagement \textbf{S}ystem (Inhaltsverwaltungssystem): Software zur Erstellung, Bearbeitung und Organisation von Inhalten der Website}
	\acro{LAMP Stack}{\textbf{L}inux \textbf{A}pache \textbf{M}ySQL \textbf{P}HP \textbf{Stack}: Programmkombination zur Entwicklung dynamischer Internetseiten}
	\acro{URL}{\textbf{U}niform \textbf{R}essource \textbf{L}ocator: identifiziert und lokalisiert eine Ressource (z.B. eine Internetseite) in Computernetzwerken}
	\acro{CI}{\textbf{C}orporate \textbf{I}dentity: Merkmale, die ein Unternehmen kennzeichnen, um es von Anderen abzugrenzen}
	\acro{HTML}{\textbf{H}yper \textbf{T}ext \textbf{M}arkup \textbf{L}anguage: Textbasierte Auszeichnungssprache zur Strukturierung digitaler Dokumente; bildet Grundlage des World Wide Web}
	\acro{responsive}{s. \acs{Responsive Design}}
	\acro{Responsive Design}{Gestalterischer und technischer Ansatz zur Erstellung von Websites, so dass diese auf Eigenschaften des jeweils benutzten Endgeräts reagieren können}
	\acro{ANSI}{\textbf{A}merican \textbf{N}ational \textbf{S}tandards \textbf{I}nstitute-Zeichencodierung: Norm zur Codierung von Schriftzeichen}
	\acro{geparst}{\textbf{parsen}: Zerlegen der Eingabe in ein zur Weiterverarbeitung geeignetes Format}
	\acro{Extension}{Erweiterung des Hauptprogrammes, um zusätzlich ben\"otigte Funktionen}
	\acro{Datenbankdump}{Teilweise oder ganze Auszüge aus einer Datenbank für die Datensicherung oder Portierung}
	\acro{getaggt}{\textbf{taggen}: Markieren wichtiger Punkte in der Entwicklungshistorie eines Softwareprojekts}
	\acro{Back-End}{Teil einer Software-Anwendung, die auf dem Server läuft und die Daten verwaltet}
	\acro{Front-End}{Anwenderseitige Oberfläche}
	\acro{Array}{Feldstrucktur, welche eine Vielzahl von gleichartigen Daten enthält}
	
\end{acronym}

