% !TEX root = Projektdokumentation.tex

% Es werden nur die Abkürzungen aufgelistet, die mit \ac definiert und auch benutzt wurden. 
%
% \acro{VERSIS}{Versicherungsinformationssystem\acroextra{ (Bestandsführungssystem)}}
% Ergibt in der Liste: VERSIS Versicherungsinformationssystem (Bestandsführungssystem)
% Im Text aber: \ac{VERSIS} -> Versicherungsinformationssystem (VERSIS)

% Hinweis: allgemein bekannte Abkürzungen wie z.B. bzw. u.a. müssen nicht ins Abkürzungsverzeichnis aufgenommen werden
% Hinweis: allgemein bekannte IT-Begriffe wie Datenbank oder Programmiersprache müssen nicht erläutert werden,
%          aber ggfs. Fachbegriffe aus der Domäne des Prüflings (z.B. Versicherung)

% Die Option (in den eckigen Klammern) enthält das längste Label oder
% einen Platzhalter der die Breite der linken Spalte bestimmt.
\begin{acronym}[WWWWWW]
	\acro{CMS}{\textbf{C}ontent \textbf{M}anagement \textbf{S}ystem (Innhaltsverwaltungssystem): Software zur Erstellung, Bearbeitung und Organisation von Inhalten der Website}
	\acro{LAMP Stack}{\textbf{L}inux \textbf{A}pache \textbf{M}ySQL \textbf{P}HP \textbf{Stack}: Programmkombination zur Entwicklung dynamischer Internetseiten}
	\acro{URL}{\textbf{U}niform \textbf{R}essource \textbf{L}ocator:  identifiziert und lokalisiert eine Ressource (z.B. eine Internetseite) in Computernetzwerken}
	\acro{CI}{\textbf{C}orporate \textbf{I}dentity : Merkmale, die ein Unternehmen kennzeichnen um es von Anderen abzugrenzen.
\end{acronym}

