% !TEX root = ../Projektdokumentation.tex
\section{Projektvorbereitung} 
\label{sec:Projektvorbereitung}
\subsection{Projektumfeld}
\label{sec:Projektumfeld}
\subsection{Vorgaben}
\label{sec:Vorgaben}
\subsection{Zielgruppenanalyse}
\label{sec:Zielgruppenanalyse}
\subsection{Stand zu Begin der Projektphase}
\label{sec:StandZuBeginDerProjektphase}

\subsection{Seitenstruktur}
\label{sec:Seitenstruktur}
\begin{figure}[ht]
	\centering
	\includegraphics[width=0.80\textwidth]{./Bilder/Men�struktur}
	\caption{Grobe Men�struktur}
	\label{fig:menuStruct}
\end{figure}
In den vorausgehenden Besprechungen wurden diverse Strukturierungsstrategien eruiert. 
In Abb. \ref{fig:menuStruct} ist eine vereinfache Darstellung der geplanten
Men�strukturierung dargestellt.


\subsection{Designentwurf}
\label{sec:Designentwurf}
Beim Entwerfen des Designs sind mehrere Faktoren mit eingeflossen. Das Hauptaugenmerk
lag hierbei darauf, dass die �ffentliche Wahrnehmung der Medienkompetenz der Schule
stark von der Gestaltung der Internetpr�senz abh�ngt.
Au�erdem sollte mit einem modernen, schlichtem Design die \acs{CI} gest�rkt, sowie m�glichst 
viele Zielgruppen erreicht werden.\\
Zur Verbesserung der Corporate Identity lag also die erste Aufgabe in der Neugestaltung des 
Logos in Verbindung mit dem Banner. Hierbei musste, wie beim Rest der Seite vorallem darauf 
geachtet werden, die Schulfarben passend einflie�en zu lassen. Insgesamt wurde zur Verwaltung 
und Bereitstellung vieler Inhalte und Informationen ein sehr �bersichtliches Design ben�tigt.
Au�erdem musste zur Abbildung von Neuigkeiten eine passende Darstellungsform gefunden 
werden.\\
Die Herausforderung bestand darin, viele Informationen und Materialien �bersichtlich in einem
modernen und responsiven Design zu vereinen.
\todo{Grobentwurf Bild anf�gen}
\input