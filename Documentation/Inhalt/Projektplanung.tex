% !TEX root = ../Projektdokumentation.tex
\section{Projektplanung} 
\label{sec:Projektplanung}

\subsection{Ressourcenplanung}
\label{sec:Ressourcenplanung}
Die Projektarbeit entstand sowohl in den Räumlichkeiten und PCs 
des Berufschulzentrums, sowie an den privaten Rechner der Autoren.\\
Eine grobe Zeitplanung, welche im Vorfeld erstellt wurde befindet sich in Tab. \ref{tab:Zeitplan}.
\tabelle{Zeitplan}{tab:Zeitplan}{Zeitplan.tex}

\subsubsection{Technische Ressourcen}
\label{sec:TechnischeRessourcen}
Als \acs{CMS} wurde TYPO3 6.2\footnote{\href{www.typo3.org}{www.typo3.org}} 
eingesetzt, da hier auch eine komplexere 
Rechteverwaltung möglich ist und sich auch mehrsprachige Projekte
gut realisieren lassen. Zudem ist dieses System quelloffen und verfügt
über eine große aktive Community.\\
Weiterhin wurde der \acs{LAMP Stack} verwendet - eine Programmkombination, 
die sich sehr gut in Kombination mit TYPO3 bewährt hat.\\
Das Grafikdesign wurde mit Adobe Photoshop 
CS3\footnote{\href{http://www.adobe.com/de/products/photoshop.html}{www.adobe.com}}, 
einem der umfangreichsten Bildbearbeitungsprogrammen, erstellt.\\
Für die Entwicklung wurde uns von der KOMSA AG\footnote{\href{https://komsa.com/de/start/}{www.komsa.com}}
ein Server bereitgestellt, sowie eine \acs{URL} (\url{www.neu_industrieschule.de}) eingerichtet.\\
Um das gemeinsame arbeiten an einem Projekt zu erleichtern wurde der Quellcode mit 
git\footnote{\href{https://git-scm.com/}{www.git-scm.com}} verwaltet.
Weitere Software, die während der Projektarbeit und zur Dokumentation eingesetzt wurde: \LaTeX, Notepad++, 
\todo{weitere Programme}

\subsubsection{Personalplanung}
\label{sec:Personalplanung}
Zu Begin des Projektes musste hier etwas improvisiert werden, da Marcel Hoffman, der die übrigen
Autoren in der Vorbereitungsphase unterstützt hat durch Krankheit ausgefallen ist. Nachfolgend sind
alle Projektmitarbeiter mit ihren Aufgabengebieten abfallend nach Priorisierung sortiert aufgelistet:
\begin{itemize}
\item Norman Paschke: Projektleitung, Entwicklung(HTML, CSS, TYPO3 Integration), Deploy
\item Johannes Thoms: Dokumentation, Präsentation, Entwicklung
\item Sebastian Fritze: Entwicklung(PHP), Deploy
\item Michael Toma: Design, Dokumentation
\end{itemize}

\subsection{Seitenstruktur}
\label{sec:Seitenstruktur}
\begin{figure}[ht]
	\centering
	\includegraphics[width=0.80\textwidth]{./Bilder/Menüstruktur}
	\caption{Grobe Menüstruktur}
	\label{fig:menuStruct}
\end{figure}
In den vorausgehenden Besprechungen wurden diverse Strukturierungsstrategien eruiert. 
In Abb. \ref{fig:menuStruct} ist eine vereinfache Darstellung der geplanten
Menüstrukturierung dargestellt.


\subsection{Designplanung}
\label{sec:Designplanung}
Bei der Planung des Designs spielten viele Faktoren eine Rolle. Dabei sollte die Medienkompetenz 
der Schule wiedergegeben werden, um sie passend zu positionieren. Außerdem sollte mit einem 
modernen, responsiven Design die Coporate Identity gestärkt, sowie möglichst viele Zielgruppen 
erreicht werden.\\
Zur Verbesserung der Corporate Identity lag also die erste Aufgabe in der Neugestaltung des 
Logos in Verbindung mit dem Banner. Hierbei musste, wie beim Rest der Seite vorallem darauf 
geachtet werden, die Schulfarben passend einfließen zu lassen. Insgesamt wurde zur Verwaltung 
und Bereitstellung vieler Inhalte und Informationen ein sehr übersichtliches Design benötigt.
Außerdem musste zur Abbildung von Neuigkeiten eine passende Darstellungsform gefunden 
werden.\\
Die Herausforderung bestand darin, viele Informationen und Materialien übersichtlich in einem
modernen und responsiven Design zu vereinen.   
\todo{Grobentwurf Bild anfügen}
\input