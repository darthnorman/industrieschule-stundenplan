% !TEX root = ../Projektdokumentation.tex
\selectlanguage{english}
\section{Introduction}
\label{sec:Introduction}

The following documentation outlines the process of the project, which was carried out by the authors
in the framework of their IT specialist education.

\subsection{Project Description}
\label{sec:ProjectDescription}
The purpose of the project was to create and realize an entirely new homepage design for the Industrieschule Chemnitz. For 
that matter the authors analyzed in precedent meetings with their teachers and the principal the current state of the school's
 internet presence and defined the frame requirements of the completely new website.

\subsection{Previous Situation}
\label{sec:PreviousSituation}
The previous homepage (see Fig. \ref{fig:pageOld}) was based on a design and technology of the year 2002. Althought it was certainly well made for its time
it became obsolete after over one decade.
\begin{figure}[ht]
	\centering
	\includegraphics[width=0.80\textwidth]{./Bilder/oldpage.jpg}
	\caption{Main menue of the previous school homepage}
	\label{fig:pageOld}
\end{figure}
The main points of criticism where the missing ease of use; i.e. the old site was not responsive and barrier-free. Also 
the missing content management system complicated administration and updating the page. Furthermore the logical 
menu structure needed revision. 

\subsection{Project Objective}
\label{sec:ProjectObjective}
It has been decided in advance that the website should be equipped with a stable updateable backend. 
The appearance of the new website was decided to be plain and functional according to the \textit{Modern UI} 
by Microsoft \cite{METRO} and fulfill today's requirements for comfort, be responsive and barrier-free. 
A further challenge is the visualisation of the timetables. The schools timetables are 
generated by an external program and need to be transformed into HTML.
\selectlanguage{ngerman}