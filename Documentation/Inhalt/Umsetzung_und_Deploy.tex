% !TEX root = ../Projektdokumentation.tex
\section{Umsetzung und Deployment} 
\label{sec:Umsetzung und Deployment}

\subsection{Entwicklung Stundenplan}
\label{sec:Entwicklung Stundenplan}
Die Schulstundenpl\"ane wurden bisher durch ein Programm erstellt, welches 
Dateien im txt-Format generiert. Diese \acs{ANSI}-kodierten Dateien enthalten, 
durch Tabulatoren getrennt, die Auflistung der R\"aume, F\"acher und Lehrer.
Da dieses Programm weiter durch den Auftraggeber genutzt werden soll, war 
es notwendig eine Konvertierung der Rohdaten in \acs{HTML} vorzunehmen.\\
Durch die M\"oglichkeit diese Dateien im Backend des \acs{CMS} hochzuladen, wurde eine 
Ordnerstruktur eingef\"uhrt, die diese Dateien nach Bl\"ocken trennt. 
Zus\"atzlich vereinfacht diese Struktur die Darstellung der Bl\"ocke und Klassen. Die Programmierung 
des Plugins zur Darstellung der Stundenpl\"ane begann mit dem Auslesen der Verzeichnisse, um einen 
\"uberblick \"uber alle verf\"ugbaren Dateien zu gewinnen und eine Auswahl f\"ur den Nutzer zur Verf\"ugung 
zu stellen. Nachdem der Nutzer den gew\"unschten Plan ausw\"ahlt, wird die entsprechende Datei 
ausgelesen, \acs{geparst}, in eine \acs{Array}struktur gebracht, um anschlie\ss{}end in Tabellenform angezeigt 
zu werden. Die Anzeige erfolgt \"uber Fluid, eine Extension von TYPO3 um die Gestaltung von Templates 
zu vereinfachen und um logische Operationen zu erweitern.

\subsection{Deployment}
\label{sec:Deployment}
Neben Ersteinrichtung des Servers und Installation von ben\"otigten \acs{Extension}s wurde ein 
\acs{Datenbankdump} mittels phpMyAdmin eingespielt. Um die Stundenplan-\acs{Extension} einzubinden,
musste diese zun\"achst \acs{getaggt} werden, bevor sie im Anschluss aufgespielt wurde.

\subsection{Redakteuersdokumentation}
\label{sec:Redakteuersdokumentation}
Um den Redakteuren, welche die Seite weiter betreuen, die Arbeit mit dem \acs{CMS} zu erleichtern, 
wurde eine technische Dokumentation angefertigt. Diese stellt einen Leitfaden zur Handhabung des 
\acs{Back-End}s zur Verf\"ugung und gew\"ahleistet ein st\"orungsfreies Verwalten neuer und alter Inhalte.