% !TEX root = ../Projektdokumentation.tex
\section{Umsetzung und Deployment} 
\label{sec:Umsetzung und Deployment}

\subsection{Skizzen und Screen-Design}
\label{sec:Skizzen und Screen-Design}
Jede Idee hat ihren Anfang auf einem Stück Papier. Wireframes sind unabdingbar, sie stellen die wichtigsten Funktionen einer 
Seite dar und dienen als Vorlage für das spätere Screen-Design. In Photoshop wurden daraufhin in mehreren Iterationen Screen-Designs
erstellt. Der finale Entwurf bildet dann die Basis für einen Prototyp.
\todo{Verweis auf Entwürfe?}

\subsection{Prototyping mit HTML und CSS}
\label{sec:Prototyping mit HTML und CSS}
Bevor das Design in TYPO3 integriert werden kann, ist es wichtig einen statischen Prototypen in HTML und CSS zu schreiben. 
In diesem Klickdummy können Fehler im Design rasch gefunden und entsprechende Lösungen entwickelt werden. Im Browser kann 
man zudem schnell improvisieren, sodass Probleme bei der User Experience - besonders im mobilen Bereich - oder der 
Barrierefreiheit schneller gelöst werden können als das in einem grafischen Entwurf der Fall ist. Sobald der Prototyp final ist,
kann damit begonnen werden, ihn mit dynamischen Inhalten zu befüllen.

\subsection{Lokale Umgebung einrichten}
\label{sec:Lokale Umgebung einrichten}
Damit keiner auf der Live-Umgebung direkt arbeitet, ist es notwendig eine geeignete Entwicklungsumgebung lokal zu errichten.
Darauf kann jeder Entwickler seine Arbeit unabhängig von anderen Entwicklern durchführen. Mit XAMPP wurde bei jedem eine lokale 
TYPO3-Installation aufgesetzt, auf der gearbeitet werden konnte. XAMMP richtet einen lokalen Apache-Server mit MySQL-Anbindung 
und einer aktuellen PHP-Version ein. Darauf kann dann eine TYPO3-Umgebung aufgesetzt werden.

\subsection{TYPO3 Integration}
\label{sec:TYPO3 Integration}
Eine nackte TYPO3-Installation und ein Prototyp allein bilden noch keine Webseite. Deswegen ist es notwendig, den erstellten 
Prototyp, der das Basistemplate der neue Website darstellt, mit dynamischen Inhalten zu befüllen. In TYPO3 wird dies durch 
TypoScript und der Template-Engine Fluid erreicht. Der Prototyp wird dabei in ein Fluid-Template überführt, in dem sogenannte 
Marker mit Inhalten ersetzt werden. Solche Inhalte sind z. B. Menüs, im Back-End angelegte Inhaltselemente, Header- und 
Footer-Bereiche der Seite.\\
Zur Integration gehört auch das Installieren von Extensions und die Einbindung deren Konfigurationsdateien. So wird 
beispielsweise die Extension realURL dazu benutzt, die URLs in ein für Menschen lesbares Format zu bringen.

\subsection{Entwicklung Stundenplan}
\label{sec:Entwicklung Stundenplan}
Die Schulstundenpl\"ane wurden bisher durch ein Programm erstellt, welches 
Dateien im txt-Format generiert. Diese \acs{ANSI}-kodierten Dateien enthalten, 
durch Tabulatoren getrennt, die Auflistung der R\"aume, F\"acher und Lehrer.
Da dieses Programm weiter durch den Auftraggeber genutzt werden soll, war 
es notwendig eine Konvertierung der Rohdaten in \acs{HTML} vorzunehmen.\\
Durch die M\"oglichkeit diese Dateien im Backend des \acs{CMS} hochzuladen, wurde eine 
Ordnerstruktur eingef\"uhrt, die diese Dateien nach Bl\"ocken trennt. 
Zus\"atzlich vereinfacht diese Struktur die Darstellung der Bl\"ocke und Klassen. Die Programmierung 
des Plugins zur Darstellung der Stundenpl\"ane begann mit dem Auslesen der Verzeichnisse, um einen 
\"uberblick \"uber alle verf\"ugbaren Dateien zu gewinnen und eine Auswahl f\"ur den Nutzer zur Verf\"ugung 
zu stellen. Nachdem der Nutzer den gew\"unschten Plan ausw\"ahlt, wird die entsprechende Datei 
ausgelesen, \acs{geparst}, in eine \acs{Array}struktur gebracht, um anschlie\ss{}end in Tabellenform angezeigt 
zu werden. Die Anzeige erfolgt \"uber Fluid, eine Extension von TYPO3 um die Gestaltung von Templates 
zu vereinfachen und um logische Operationen zu erweitern.

\subsection{Entwicklungen unter Versionskontrolle stellen}
\label{sec:Entwicklungen unter Versionskontrolle stellen}
Um eine einheitliche Versionierung und einen kontrollierten Entwicklungsprozess sicherstellen zu können, wurde die Extension
Stundenplan unter Versionskontrolle durch Git gestellt. Dadurch können parallele Arbeiten schnell und sicher zusammengeführt 
und bei Problemen die Änderungshistorie durchsucht und ggf. alte Stände zurückgeholt werden. Git stellt alle Dateien in einem 
Repository zusammen. Darunter befinden sich neben dem Quellcode zur Darstellung der Stundenpläne auch das Fluid-Template sowie 
alle Bilder, CSS-, und JavaScript-Dateien.\\
Entwickler können ihre Änderungen, die sie lokal gemacht haben, an diesen zentralen Ort senden und durch einen Merge werden 
die Änderungen mit der bisherigen Version zu einer neuen Version verschmolzen. Das Repository befindet ist derzeit unter Github 
(https://github.com/darthnorman/industrieschule-stundenplan), ist öffentlich zugänglich und kann von jedem heruntergeladen 
und geforkt werden.

\subsection{Deployment}
\label{sec:Deployment}
Neben Ersteinrichtung des Servers und Installation von ben\"otigten \acs{Extension}s wurde ein 
\acs{Datenbankdump} mittels phpMyAdmin eingespielt. Um die Stundenplan-\acs{Extension} einzubinden,
musste diese zun\"achst \acs{getaggt} werden, bevor sie im Anschluss aufgespielt wurde.

\todo{Testing?}

\subsection{Redakteuersdokumentation}
\label{sec:Redakteuersdokumentation}
Um den Redakteuren, welche die Seite weiter betreuen, die Arbeit mit dem \acs{CMS} zu erleichtern, 
wurde eine Redakteuersdokumentation angefertigt. Diese umfasst neben den grundlegenden Kenntnissen im Umgang mit TYPO3 
einen Leitfaden zur Handhabung des \acs{Back-End}s das Erstellen von Seiten und Inhalten wie Bilder, Texte, Neuigkeiten 
und Stundenpläne. Weiterhin wird die Rechte- und Rollenverteilung des Portals offen gelegt und auf sonstige Eigenheiten 
des Portals eingegangen.
\todo{Doku in den Anhang}