% !TEX root = ../Projektdokumentation.tex
\section{Conclusion} 
\label{sec:Conclusion}
At the end of the project period all of the planned tasks have been finished. 
This includes the preparation of the server to host the website, the configuration 
of the \acs{Back-End} software, finishing and applying the designs and the creation 
of the presentation and documentation.\\
\vspace{10ex}
In addition the group managed to fulfil the optional task to design the homepage \acs{responsive}.

\subsection{Problems}
During the project several problems arose. One of them was the absence of a team member for 
reasons of health, which was not calculated in the time schedule, but could be 
solved by a good communication between the remaining team members as well as the an 
outside estimate of the planned tasks. A comparison of the targeted timetable
and actually required times can be seen in \ref{tab:SollIstVergleich}. The second 
problem emerged when we were deploying test data on the server designated to host 
the website. It became clear that the provided credentials for the communication 
with the server were erroneous. After acquiring the correct credentials, the hosting 
server was changed, which meant for us to deploy and configure everything from scratch.

\tabelle{Comparison of target timetable and the time actually required for the specific tasks}{tab:SollIstVergleich}{SollIstVergleich.tex}

\subsection{Lessons Learned}
Despite the afore mentioned problems, or even because of them, the project did have 
some positive effects. The team learned quickly to assign tasks efficiently and worked 
effectively to solve the given tasks. The communication between the team members has 
significantly improved compared to the time preparing the project. The dealing with 
partly new software lead to new or advanced insights which may be useful in upcoming 
projects. All in all the project was a success, not only in finishing the website, 
but also in gaining experience in organizing and working as a team.


\subsection{Acknowledgment}
The authors like to thank Mr. Ciborra, Mr. M\"uller and Principal Hunger 
for the opportunity to work on this interesting and challenging project 
and for their support during its implementation.\\
They also like to give thanks to Mr. Weihe for the good grades he is going 
to give them now.

% \begin{itemize}
	% \item Wurde das Projektziel erreicht und wenn nein, warum nicht?
	% \item Ist der Auftraggeber mit dem Projektergebnis zufrieden und wenn nein, warum nicht?
	% \item Wurde die Projektplanung (Zeit, Kosten, Personal, Sachmittel) eingehalten oder haben sich Abweichungen ergeben und wenn ja, warum?
	% \item Hinweis: Die Projektplanung muss nicht strikt eingehalten werden. Vielmehr sind Abweichungen sogar als normal anzusehen. Sie müssen nur vernünftig begründet werden (\zB durch Änderungen an den Anforderungen, unter-/überschätzter Aufwand).
% \end{itemize}

% \paragraph{Beispiel (verkürzt)}
% Wie in Tabelle~\ref{tab:Vergleich} zu erkennen ist, konnte die Zeitplanung bis auf wenige Ausnahmen eingehalten werden.
% \tabelle{Soll-/Ist-Vergleich}{tab:Vergleich}{Zeitnachher.tex}



% \begin{itemize}
	% \item Was hat der Prüfling bei der Durchführung des Projekts gelernt (\zB Zeitplanung, Vorteile der eingesetzten Frameworks, Änderungen der Anforderungen)?
% \end{itemize}


% \subsection{Outlook}
% \label{sec:Outlook}

% \begin{itemize}
	% \item Wie wird sich das Projekt in Zukunft weiterentwickeln (\zB geplante Erweiterungen)?
% \end{itemize}
